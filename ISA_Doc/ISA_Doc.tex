\documentclass{article}

\usepackage[letterpaper,margin=0.5in,top=1in,bottom=1in]{geometry}

\usepackage{extramarks}
\usepackage{amsmath}
\usepackage{amsthm}
\usepackage{amsfonts}
\usepackage{tikz}
	\usetikzlibrary{positioning}
\usepackage[plain]{algorithm}
\usepackage{algpseudocode}
\usepackage{enumitem}
\usepackage{fancyhdr}
	\pagestyle{fancy}
	\lhead{\hmwkAuthorName}
	% \chead{\hmwkClass\ (\hmwkClassInstructor\ \hmwkClassTime): \hmwkTitle}
	\rhead{\firstxmark}
	\lfoot{\lastxmark}
	\cfoot{\thepage}

	\renewcommand{\headrulewidth}{0.4pt}
	\renewcommand{\footrulewidth}{0.4pt}
	\setlength{\headheight}{14.5pt}
\usepackage{fontspec}
	\setmainfont{Doulos SIL}
\usepackage{booktabs}
\usepackage{lscape}
\usepackage{longtable}
\usepackage{bytefield}
\usepackage{array}
\usepackage{hyperref}

\setlength\parindent{0pt}

\newcommand{\hmwkTitle}{ISA Document}
\newcommand{\hmwkClass}{CSCI-453 Computer Architecture}
\newcommand{\hmwkAuthorName}{\textbf{Max Kipust, Drew Young}}

%
% Title Page
%

\title{
	\vspace{2in}
	\textmd{\textbf{\hmwkClass:\ \hmwkTitle}}

	\vspace{0.5in}
	\normalsize{When you wanted a Mill but you build a S.H.E.D in your backyard instead}\\
	\begin{tabular}{l}
		\normalsize{S: Student}\\
		\normalsize{H: Hell Of}\\
		\normalsize{E: Eternal}\\
		\normalsize{D: Development}\\
	\end{tabular}


	\vspace{3in}
}

\author{\hmwkAuthorName}
\date{}

% \renewcommand{\part}[1]{\textbf{\large Part \Alph{partCounter}}\stepcounter{partCounter}\\}

%
% Various Helper Commands
%

% Useful for algorithms
\newcommand{\alg}[1]{\textsc{\bfseries \footnotesize #1}}

\begin{document}

\maketitle

\pagebreak

\section{Instruction Format}

	Our instructions will be a 16-bits each, no variable-length instructions.
	To handle different configurations of operands, we will have 6 different instruction formats.
	Instructions take 0, 1 or 2 belt positions and 0 or 1 immediates.

	The instruction is split into 4 fields of sizes 2-bits, 3-bits, 3-bits and 4-bits respectively.
	Each field has a designated purpose, but may be used for opcode space if its purpose is unnecessary for a set of instructions.

	\vspace{5mm}

	\begin{bytefield}[endianness=big,bitwidth=3em,bitheight=6ex]{16}
		\bitheader[bitformatting={}]{0,7,8,10,11,13,14,15} \\
		\bitbox{2}{Field 1 \\ Opcode} &
		\bitbox{3}{Field 2 \\ Belt Position (\(B_1\))} &
		\bitbox{3}{Field 3 \\ Belt Position (\(B_2\))} &
		\bitbox{8}{Field 4 \\ Immediate (\(\textrm{Imm}\))} \\
	\end{bytefield}


\section{Address Modes}

	We have 4 address modes:
	\begin{description}[font=\sffamily\bfseries]
		\item[data] \(\textrm{Addr} = B_1 + \operatorname{sign\_ext}\left(\textrm{Imm}\right)\)

			Used by \texttt{load} and \texttt{store}.

		\item[stack] \(\textrm{Addr} = \textrm{FP} + \textrm{Imm}\)

			Used by \texttt{read\_stack} and \texttt{write\_stack}.

		\item[code] \(\textrm{Addr} = \textrm{PC} + \operatorname{sign\_ext}\left(\textrm{Imm}\right)\)

			Used by \texttt{jump}, \texttt{call}, \texttt{branch\_zero}, \texttt{branch\_neg}, \texttt{branch\_oflow} and \texttt{branch\_carry}.

		\item[long code] \(\textrm{Addr} = B_1\)

			Used by \texttt{longjump} and \texttt{longcall}.

	\end{description}

	Which address mode is used is intrinsic to each opcode, for example \texttt{store} always uses the data address mode.
	There are some instructions which are the same except for the address mode (\texttt{write\_stack} is \texttt{store} with the stack address mode), but we don't have room for the address mode selection to be a separate field.

\section{Scheduling}
	Each of our instructions are statically scheduled.
	Each instruction has a specified latency between when the instruction is issued and when the result is visible to the program.
	Our instruction list has an estimated latency, these may change when we actually implement each instruction.

\clearpage
\begin{landscape}
\section{Instruction List}

	\begin{longtable}{c c c c l l c l}
		\multicolumn{1}{c}{Field 1} &
		\multicolumn{1}{c}{Field 2} &
		\multicolumn{1}{c}{Field 3} &
		\multicolumn{1}{c}{Field 4} &
		\multicolumn{1}{c}{Mnemonic} &
		\multicolumn{1}{c}{Description} &
		\multicolumn{1}{c}{Latency} &
		\multicolumn{1}{c}{RTL} \\
		\midrule
		\texttt{00} & belt pos & belt pos & immediate & \texttt{store} & Store to data address mode & N/A &
			\(\textrm{Mem}\left[B_1 + \operatorname{sign\_ext}\left(\textrm{Imm}\right)\right] \gets B_2\) \\

		\texttt{01} & belt pos & belt pos & \texttt{0000 0000} & \texttt{add}     & Add                    & 1 &
			\(\textrm{Belt} \gets B_1 + B_2\) \\
		\texttt{01} & belt pos & belt pos & \texttt{0000 0001} & \texttt{mul}     & Unsigned Multiply      & 5 &
			\(\begin{array}{l}
				\textrm{Belt} \gets \left(B_1 * B_2\right) << 16, \\
				\textrm{Belt} \gets B_1 * B_2
			\end{array}\) \\
		\texttt{01} & belt pos & belt pos & \texttt{0000 0010} & \texttt{and}     & Bitwise and            & 1 &
			\(\textrm{Belt} \gets B_1 \mathop{\&} B_2\) \\
		\texttt{01} & belt pos & belt pos & \texttt{0000 0011} & \texttt{or}      & Bitwise or             & 1 &
			\(\textrm{Belt} \gets B_1 \mathop{|} B_2\) \\
		\texttt{01} & belt pos & belt pos & \texttt{0000 0100} & \texttt{xor}     & Bitwise exclusive or   & 1 &
			\(\textrm{Belt} \gets B_1 \oplus B_2\) \\
		\texttt{01} & belt pos & belt pos & \texttt{0000 0101} & \texttt{lshift}  & Left shift             & 1 &
			\(\textrm{Belt} \gets B_1 << B_2\) \\
		\texttt{01} & belt pos & belt pos & \texttt{0000 0110} & \texttt{rshift}  & Logical right shift    & 1 &
			\(\textrm{Belt} \gets B_1 >> B_2\) \\
		\texttt{01} & belt pos & belt pos & \texttt{0000 0111} & \texttt{arshift} & Arithmetic right shift & 1 &
			\(\textrm{Belt} \gets B_1 >>_a B_2\) \\
		\texttt{01} & belt pos & belt pos & \texttt{0000 1000} & \texttt{addc}    & Add carry              & 1 &
			\(\textrm{Belt} \gets B_1 + \operatorname{carry}\left(B_2\right)\) \\
		\texttt{01} & belt pos & belt pos & \texttt{0000 1001} & \texttt{sub}     & Subtract               & 1 &
			\(\textrm{Belt} \gets B_1 - B_2\) \\
		\texttt{01} & belt pos & belt pos & \texttt{0000 1010} & \texttt{ddump}   & Dump Data Memory       & 1 &
			 \\
		\texttt{01} & belt pos & belt pos & \texttt{0000 1011} & \texttt{sdump}   & Dump Stack Memory       & 1 &
			 \\
		\texttt{01} & & & \texttt{0000 1100} & INVALID & & \\
								& & & \(\vdots\)         &         & & \\
		\texttt{01} & & & \texttt{1111 1110} & INVALID & & \\
		\texttt{01} & belt pos & \texttt{000} & \texttt{1111 1111} & \texttt{invert}     & Bitwise inversion              & 1   &
			\(\textrm{Belt} \gets \overline{B_1}\) \\
		\texttt{01} & belt pos & \texttt{001} & \texttt{1111 1111} & \texttt{negate}     & Numeric negation               & 1   &
			\(\textrm{Belt} \gets -B_1\) \\
		\texttt{01} & belt pos & \texttt{010} & \texttt{1111 1111} & \texttt{longcall}   & Call to long code address mode & 2 &
			\(\begin{array}{l}
				\textrm{Mem}\left[\textrm{SP}\right] \gets \textrm{FP}, \\
				\textrm{Mem}\left[\textrm{SP} + 1\right] \gets \textrm{PC} + 1, \\
				\textrm{PC} \gets B_1, \\
				\textrm{FP} \gets \textrm{SP} + 2 \\
			\end{array}\) \\
		\texttt{01} & belt pos & \texttt{011} & \texttt{1111 1111} & \texttt{longjump}   & Jump to long code address mode & N/A &
			\(\textrm{PC} \gets B_1\) \\
		\texttt{01} & belt pos & \texttt{100} & \texttt{1111 1111} & \texttt{get\_carry} & Retrieve carry metadata        & 1   &
			\(\textrm{Belt} \gets \operatorname{carry}\left(B_1\right)\) \\
		\texttt{01} & & \texttt{101} & \texttt{1111 1111} & INVALID & \\
		\texttt{01} & & \texttt{110} & \texttt{1111 1111} & INVALID & \\
		\texttt{01} & \texttt{000} & \texttt{111} & \texttt{1111 1111} & \texttt{halt}    & Halt the computer       & N/A & HALT \\
		\texttt{01} & \texttt{001} & \texttt{111} & \texttt{1111 1111} & \texttt{nop}     & No operation            & N/A & \\
		\texttt{01} & \texttt{010} & \texttt{111} & \texttt{1111 1111} & \texttt{return}  & Return from function    & 2 &
			\(\begin{array}{l}
				\textrm{PC} \gets \textrm{Mem}\left[\textrm{FP} - 1\right], \\
				\textrm{FP} \gets \textrm{Mem}\left[\textrm{FP} - 2\right], \\
				\textrm{SP} \gets \textrm{FP} - 2 \\
			\end{array}\) \\
		\texttt{01} & \texttt{011} & \texttt{111} & \texttt{1111 1111} & \texttt{get\_pc} & Retrieve the current PC & 1   &
			\(\textrm{Belt} \gets \textrm{PC}\) \\
		\texttt{01} & \texttt{100} & \texttt{111} & \texttt{1111 1111} & INVALID & \\
		            & \(\vdots\)   &              &                    &         & \\
		\texttt{01} & \texttt{111} & \texttt{111} & \texttt{1111 1111} & INVALID & \\

		\texttt{10} & belt pos & \texttt{000} & immediate & \texttt{addi}     & Add with immediate                    & 1 &
			\(\textrm{Belt} \gets B_1 + \operatorname{sign\_ext}\left(\textrm{Imm}\right)\) \\
		\texttt{10} & belt pos & \texttt{001} & immediate & \texttt{muli}     & Unsigned multiply with immediate      & 5 &
			\(\begin{array}{l}
				\textrm{Belt} \gets \left(B_1 * \operatorname{sign\_ext}\left(\textrm{Imm}\right)\right) << 16, \\
				\textrm{Belt} \gets B_1 * \operatorname{sign\_ext}\left(\textrm{Imm}\right)
			\end{array}\) \\
		\texttt{10} & belt pos & \texttt{010} & immediate & \texttt{andi}     & Bitwise and with immediate            & 1 &
			\(\textrm{Belt} \gets B_1 \mathop{\&} \operatorname{sign\_ext}\left(\textrm{Imm}\right)\) \\
		\texttt{10} & belt pos & \texttt{011} & immediate & \texttt{ori}      & Bitwise or with immediate             & 1 &
			\(\textrm{Belt} \gets B_1 \mathop{|} \textrm{Imm}\) \\
		\texttt{10} & belt pos & \texttt{100} & immediate & \texttt{xori}     & Bitwise exclusive or with immediate   & 1 &
			\(\textrm{Belt} \gets B_1 \oplus \textrm{Imm}\) \\
		\texttt{10} & belt pos & \texttt{101} & immediate & \texttt{lshifti}  & Left shift with immediate             & 1 &
			\(\textrm{Belt} \gets B_1 << \textrm{Imm}\) \\
		\texttt{10} & belt pos & \texttt{110} & immediate & \texttt{rshifti}  & Logical right shift with immediate    & 1 &
			\(\textrm{Belt} \gets B_1 >> \textrm{Imm}\) \\
		\texttt{10} & belt pos & \texttt{111} & immediate & \texttt{arshifti} & Arithmetic right shift with immediate & 1 &
			\(\textrm{Belt} \gets B_1 >>_a \textrm{Imm}\) \\

		\texttt{11} & belt pos & \texttt{000} & immediate & \texttt{branch\_zero}  & Branch if zero to code address mode & N/A &
			\(\begin{array}{l}
				\textrm{If } B_1 = 0 \textrm{,} \\
				\textrm{then } \textrm{PC} \gets \textrm{PC} + \operatorname{sign\_ext}\left(\textrm{Imm}\right)
			\end{array}\) \\
		\texttt{11} & belt pos & \texttt{001} & immediate & \texttt{branch\_neg}   & Branch if negative to code address mode & N/A &
			\(\begin{array}{l}
				\textrm{If } B_1\left[15\right] = 1 \textrm{,} \\
				\textrm{then } \textrm{PC} \gets \textrm{PC} + \operatorname{sign\_ext}\left(\textrm{Imm}\right)
			\end{array}\) \\
		\texttt{11} & belt pos & \texttt{010} & immediate & \texttt{branch\_oflow} & Branch on overflow to code address mode & N/A &
			\(\begin{array}{l}
				\textrm{If } \operatorname{overflow}\left(B_1\right) = 1 \textrm{,} \\
				\textrm{then } \textrm{PC} \gets \textrm{PC} + \operatorname{sign\_ext}\left(\textrm{Imm}\right)
			\end{array}\) \\
		\texttt{11} & belt pos & \texttt{011} & immediate & \texttt{branch\_carry} & Branch on carry to code address mode    & N/A &
			\(\begin{array}{l}
				\textrm{If } \operatorname{carry}\left(B_1\right) = 1 \textrm{,} \\
				\textrm{then } \textrm{PC} \gets \textrm{PC} + \operatorname{sign\_ext}\left(\textrm{Imm}\right)
			\end{array}\) \\
		\texttt{11} & belt pos & \texttt{100} & immediate & \texttt{write\_stack}  & Write to stack position & N/A &
			\(\textrm{Mem}\left[\textrm{FP} + \operatorname{sign\_ext}\left(\textrm{Imm}\right)\right] \gets B_1\) \\
		\texttt{11} & belt pos & \texttt{101} & immediate & \texttt{load}          & Load from memory        & 1   &
			\(\textrm{Belt} \gets \textrm{Mem}\left[B_1 + \operatorname{sign\_ext}\left(\textrm{Imm}\right)\right]\) \\
		\texttt{11} & & \texttt{110} & & INVALID & \\
		\texttt{11} & \texttt{000} & \texttt{111} & immediate & \texttt{constant}    & Load immediate            & 1   &
			\(\textrm{Belt} \gets \operatorname{sign\_ext}\left(\textrm{Imm}\right)\) \\
		\texttt{11} & \texttt{001} & \texttt{111} & immediate & \texttt{upper}       & Load upper immediate      & 1   &
			\(\textrm{Belt} \gets \textrm{Imm} << 8\) \\
		\texttt{11} & \texttt{010} & \texttt{111} & immediate & \texttt{call}        & Call to code address mode & 2 &
			\(\begin{array}{l}
				\textrm{Mem}\left[\textrm{SP}\right] \gets \textrm{FP}, \\
				\textrm{Mem}\left[\textrm{SP} + 1\right] \gets \textrm{PC} + 1, \\
				\textrm{PC} \gets \textrm{PC} + \operatorname{sign\_ext}\left(\textrm{Imm}\right), \\
				\textrm{FP} \gets \textrm{SP} + 2 \\
			\end{array}\) \\
		\texttt{11} & \texttt{011} & \texttt{111} & immediate & \texttt{jump}        & Jump to code address mode & 2 &
			\(\textrm{PC} \gets \textrm{PC} + \operatorname{sign\_ext}\left(\textrm{Imm}\right)\) \\
		\texttt{11} & \texttt{100} & \texttt{111} & immediate & \texttt{read\_stack} & Read from stack position  & 1   &
			\(\textrm{Belt} \gets \textrm{Mem}\left[\textrm{FP} + \textrm{Imm}\right]\) \\
		\texttt{11} & \texttt{101} & \texttt{111} & immediate & \texttt{alloca}      & Allocate stack space      & N/A &
			\(\textrm{SP} \gets \textrm{SP} + \textrm{Imm}\) \\
		\texttt{11} & \texttt{110} & \texttt{111} & & INVALID & \\
		\texttt{11} & \texttt{111} & \texttt{111} & & INVALID & \\
	\end{longtable}
\end{landscape}

\section{Memory Space}

	We will split memory up into 3 sections: Code, Stack, and Heap.
	This will allow us to enforce the separation of the different sections and will reduce hazards.

	Each section will have an address size of 16 bits, and each piece of data will also be 16 bits.

	\subsection*{Code}

		The code section will only be accessible by fetch.
		This means that self modifying code is not possible but we think that this is a reasonable tradeoff.
		Any instruction that uses the code or long code address modes will be addressing into the code section.

	\subsection*{Stack}

		The stack will only be directly accessible by \texttt{read\_stack} and \texttt{write\_stack}.
		The \texttt{alloca} instruction will be used to allocate stack space and function calls and returns will use the stack to maintain the frame and stack pointers and return address.
		We will enforce stack safety. If any instruction attempts to access stack space that it isn't supposed to be able to access, we will throw an error.
		Any instruction that uses the stack address mode will be able to access the stack section.

	\subsection*{Heap}

		The heap will be directly accessed by \texttt{load} and \texttt{store}.
		Any instruction that uses the data address mode will be able to access the heap section.

\section{Program Visible Registers}

	The general purpose registers will be represented as a belt.
	Each instruction implicitly puts the result of the operation at the beginning of the belt and shifts the rest of the values down.
	Any value that overflows off the end of the belt is lost.

	We will also have a stack pointer, frame pointer and program counter.
	These registers aren't directly accessible, but they are visible through several instructions.


\end{document}
